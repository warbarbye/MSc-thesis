\newpage
\section{Obliczenia ewolucyjne}
\label{sec:ewol}
\subsection{Metafora biologiczna}

    Obliczenia ewolucyjne należą do licznej rodziny algorytmów mimetycznych i bazują na
    procesie ewolucji biologicznej, która została po raz pierwszy wyrażona przez Ch. Darwina w XIX wieku \source. Z tego względu w celu pełnego zrozumienia logiki tych algorytmów warte wydaje się wprowadzenie i zdefiniowanie podstawowych pojęć, które stosuje ta dziedzina. 
    Nie jest to bynajmniej konieczne. Część badaczy przyjmuje nomenklaturę biologii ewolucyjnej jedynie za pewną konwencję notacyjną, traktując algorytmy ewolucyjne jako instancje metod przeszukiwania losowego \source. Pogląd taki wydaje się uzasadniony, gdy zwróci się uwagę na fakt, że obliczenia ewolucyjne poświęcone są optymalizacji w rozumieniu tego pojęcia zgodnie z definicją podaną w rozdziale \ref{sec:optim}, a procesy ewolucyjne -- adaptacji \source.
    Ponadto wraz z pojawieniem się i popularyzacją pojęcia \textit{metaheurystyk} bazowanie wyłącznie na nośnej hipotezie przyczyniło się do powstania wielu redundantnych heurystyk optymalizacyjnych i badań im poświęconych \source.
    W związku z powyższym zdefiniowane poniżej pojęcia nie powinny być traktowane jako ścisłe odpowiedniki pojęć biologii ewolucyjnej, a jako wspólna i luźno powiązana baza pojęciowa różnych paradygmatów obliczeń ewolucyjnych.
    
\subsection{Algorytm ewolucyjny}

    Podobnie jak w rozdziale \source algorytm ewolucyjny może zostać sformułowany na wysokim poziomie abstrakcji i stanowi on specyfikację algorytmu optymalizacyjnego $AlgOpt$. W nawiązaniu do analogii biologicznej punkt lub zbiór punktów startowych $X_0$ będzie tożsamy \textbf{populacji} początkowej $P_0$. Populacja składa się z \texbf{osobników bazowych}, które zostają poddane co najmniej jednemu operatorowi wariacyjnemu, tj. operatorowi mutacji $\Pi_{m}$ lub rekombinacji $\Pi_{r}$. Wynikiem działania tych operatorów jest zbiór \textbf{osobników potomnych}, który oznaczany jest symbolem $O_{t}$, gdzie $t$ jest licznikiem \textbf{pokolenia}. Następnie zbiór $O_{t}$ jest poddawany działaniu operatora selekcji $\Pi_{s}$, który na podstawie \textbf{przystosowania} wybiera osobniki, które utworzą nową populację osobników bazowych $P_{t+1}$. Iteracyjny proces ewolucji podobnie jak każdy algorytm optymalizacyjny sterowany jest przez zbiór parametrów $\Sigma$, który
    ze względu na mnogość stosowanych operatorów oraz charakter samych parametrów może zostać podzielony na dwa rozłączne zbiory -- zbiór parametrów wewnętrznych $\Sigma_{en}$ (ang. \textit{endogenous parameters}) oraz zbiór parametrów zewnętrznych $\Sigma_{ex}$ (ang. \textit{exogenous parameters}). Pierwszy z nich składa się z parametrów algorytmu, które zostają poddane procesowi ewolucji w ramach mechanizmu \textbf{samoadaptacji}, a drugi zawiera te parametry, które są stałe w trakcie działania algorytmu. Celem stosowania operatorów wariacyjnych i selekcji jest zarządzanie rozrzutem populacji w taki sposób, aby zapewnić odpowiedni balans między \textbf{eksploatacją}, a \textbf{eksploracją} przestrzeni przeszukiwań. Zbyt duży rozrzut populacji będzie prowadzić do rozbieżności generowanego ciągu populacji $P_1, P_2, \dots$, a z kolei nadmierna eksploatacja -- do przedwczesnej zbieżności algorytmu. Należy nadmienić, że do tej pory nie została opracowana powszechnie akceptowana definicja tych pojęć \source. Ponadto poza założeniem, że operator wariacyjny $\Pi_v$ zwiększa, a operator selekcji $\Pi_s$ zmniejsza rozrzut populacji, nie istnieją inne ogólne wytyczne dotyczące działania tych operatorów \source. Wytyczne takie pojawiają się wyłącznie w kontekście konkretnych odmian algorytmów ewolucyjnych, które są tematem podrozdziału \source. 
    Innymi słowy, algorytm ewolucyjny może zostać przedstawiony jako następująca krotka:
    
    \begin{equation}
         EvolAlgOpt & := \left(P_0, \Pi_s, \Pi_v,  \kappa, \Sigma_s, \Sigma_{v} \right) \\
    \end{equation}
    w której:
    \begin{align*}
         \Pi_{v}:\; &  (P, \Sigma_{v}) \rightarrow O \\
         \Pi_{s}:\; &  (O, \Sigma_{s}) \rightarrow P \\
         \Sigma & = \Sigma_{v} \cup \Sigma_{s} \\
         \Pi & = \Pi_{v} \cup \Pi_{s}. \\
    \end{align*}
    
Natomiast sama logika działania algorytmu przebiega ze schematem pokazanym na wydruku \source. Wszystkie algorytmy ewolucyjne bazują dokładnie na tej samej sekwencji operacji. Różnice między poszczególnymi algorytmami ewolucyjnymi dotyczą sposobu reprezentacji osobników, tj. ich \textbf{genotypu}, oraz realizacji operatorów wariacyjnych oraz operatora selekcji. O ile przedstawiona logika działania algorytmów jest do siebie zbliżona, tak różnice w reprezentacji i jej konsekwencjach prowadzą do rodzin algorytmów na tyle odmiennych, że dają się one ująć w osobnych paradygmatach. 
\begin{algorithm}[h]
\caption{CMA-ES}
\label{alg-evol}
\begin{algorithmic}[1]
\STATE inicjalizuj $(P_1, \Sigma_{s}, \Sigma_{v})$
\STATE $t \gets 1$
\WHILE{$\kappa$ nie jest spełnione}
    \STATE $O_{t} \gets \Pi_{v}(P_{t})$
    \STATE $eval(O_{t})$
    \STATE $P_{t+1} \gets \Pi_{s}(O_{t}, P_{t})$
    \STATE $t \gets t + 1$
\ENDWHILE
\end{algorithmic}
\end{algorithm}

\subsection{Paradygmaty obliczeń ewolucyjnych}
\subsubsection{Algorytmy genetyczne}
    Jednym z najprostszych i najczęściej przytaczanych paradygmatów algorytmów ewolucyjnych są algorytmy genetyczne (GA, ang. \textit{genetic algorithm}) wprowadzone przez Hollanda \source. Paradygmat ten charakteryzuje się binarną reprezentacją osobników, tj.: 
    \begin{equation*}
        P_{t} \subset \{0, 1\}^{n}
    \end{equation*}q
    i realizacją operatorów wariacyjnych w postaci mutacji oraz krzyżowania. Dla kanonicznego algorytmu genetycznego (SGA, ang. \textit{Simple Genetic Algorithm})
    operator wariacyjny jest realizowany przez operator krzyżowania oraz mutacji.
    Krzyżowanie osobników odbywa się poprzez losowe połączenie ich w pary 
    operator mutacji jest realizowany jako funkcja dokonująca inwersji każdego bitu osobnika z prawdopodobieństwem $p_m$.  
\subsubsection{Programowanie ewolucyjne}
\subsubsection{Programowanie genetyczne}
\subsubsection{Strategie ewolucyjne}

