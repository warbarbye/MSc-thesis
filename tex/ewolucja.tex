\newpage
\section{Obliczenia ewolucyjne}
\label{sec:ewol}
\subsection{Metafora biologiczna}

    Obliczenia ewolucyjne należą do licznej rodziny algorytmów mimetycznych i bazują na
    procesie ewolucji biologicznej, która została po raz pierwszy wyrażona przez Ch. Darwina w XIX wieku {source}. Z tego względu w celu pełnego zrozumienia logiki tych algorytmów warte wydaje się wprowadzenie i zdefiniowanie podstawowych pojęć, które stosuje ta dziedzina. 
    Nie jest to bynajmniej konieczne. Część badaczy przyjmuje nomenklaturę biologii ewolucyjnej jedynie za pewną konwencję notacyjną, traktując algorytmy ewolucyjne jako instancje metod przeszukiwania losowego {source}. Pogląd taki wydaje się uzasadniony, gdy zwróci się uwagę na fakt, że obliczenia ewolucyjne poświęcone są optymalizacji w rozumieniu tego pojęcia zgodnie z definicją podaną w rozdziale \ref{sec:optim}, a procesy ewolucyjne -- adaptacji {source}.
    Ponadto wraz z pojawieniem się i popularyzacją pojęcia \textit{metaheurystyk} bazowanie wyłącznie na nośnej hipotezie przyczyniło się do powstania wielu redundantnych heurystyk optymalizacyjnych i badań im poświęconych {source}.
    W związku z powyższym zdefiniowane poniżej pojęcia nie powinny być traktowane jako ścisłe odpowiedniki pojęć biologii ewolucyjnej, a jako wspólna i luźno powiązana baza pojęciowa różnych paradygmatów obliczeń ewolucyjnych.
    
\subsection{Algorytm ewolucyjny}

    Podobnie jak w rozdziale {source} algorytm ewolucyjny może zostać sformułowany na wysokim poziomie abstrakcji i stanowi on specyfikację algorytmu optymalizacyjnego $AlgOpt$. W nawiązaniu do analogii biologicznej punkt lub zbiór punktów startowych $X_0$ będzie tożsamy \textbf{populacji} początkowej $P_0$. Populacja składa się z \texbf{osobników bazowych}, które zostają poddane co najmniej jednemu operatorowi wariacyjnemu, tj. operatorowi mutacji $\Pi_{m}$ lub rekombinacji $\Pi_{r}$. Wynikiem działania tych operatorów jest zbiór \textbf{osobników potomnych}, który oznaczany jest symbolem $O_{t}$, gdzie $t$ jest licznikiem \textbf{pokolenia}. Następnie zbiór $O_{t}$ jest poddawany działaniu operatora selekcji $\Pi_{s}$, który na podstawie \textbf{przystosowania} wybiera osobniki, które utworzą nową populację osobników bazowych $P_{t+1}$. Iteracyjny proces ewolucji podobnie jak każdy algorytm optymalizacyjny sterowany jest przez zbiór parametrów $\Sigma$, który
    ze względu na mnogość stosowanych operatorów oraz charakter samych parametrów może zostać podzielony na dwa rozłączne zbiory -- zbiór parametrów wewnętrznych $\Sigma_{en}$ (ang. \textit{endogenous parameters}) oraz zbiór parametrów zewnętrznych $\Sigma_{ex}$ (ang. \textit{exogenous parameters}). Pierwszy z nich składa się z parametrów algorytmu, które zostają poddane procesowi ewolucji w ramach mechanizmu \textbf{samoadaptacji}, a drugi zawiera te parametry, które są stałe w trakcie działania algorytmu. 
    Innymi słowy, algorytm ewolucyjny może zostać przedstawiony jako następująca krotka:
    
    \begin{equation}
         EvolAlgOpt & := \left(P_0, \Pi_s, \Pi_v,  \kappa, \Sigma_s, \Sigma_{v} \right) \\
    \end{equation}
    w której:
    \begin{align*}
         \Pi_{v}:\; &  (P, \Sigma_{v}) \rightarrow O \\
         \Pi_{s}:\; &  (O, \Sigma_{s}) \rightarrow P \\
         \Sigma & = \Sigma_{v} \cup \Sigma_{s} \\
         \Pi & = \Pi_{v} \cup \Pi_{s}. \\
    \end{align*}
    
Natomiast sama logika działania algorytmu przebiega z następującym schematem: 

\begin{algorithm}[h]
\caption{CMA-ES}
\label{alg-CMA-ES}
\begin{algorithmic}[1]
\STATE $t \gets 1$
\STATE initialize$(\wek{m}^1,\sigma^1, C^1)$
\STATE $\wek{p}_c^1 \gets \wek{0}$, $\wek{p}_\sigma^1 \gets \wek{0}$
\WHILE{!stop}
   \FOR{$i=1$ \TO $\lambda$}
      \STATE $ \wek{d}_i^t \sim N(\wek{0}, \mat{C}^t) $
      \STATE $\wek{x}_i^t=\wek{m}^{t} + \sigma^t \wek{d}_i^t $
      \STATE evaluate $(\wek{x}_i^t)$
   \ENDFOR
   \STATE sort $ \left(\{ \wek{x}_i^t \} \right) $
  
   \STATE $\wek{\Delta}^{t} \gets \sum_{i=1}^\mu w_i \wek{d}_i^t $
   \STATE $\wek{m}^{t+1} \gets \wek{m}^{t+1} + \sigma^t \wek{\Delta}^{t} $
   \STATE $\wek{p}_c^{t+1} \gets (1-c_p)\wek{p}_c^t + \sqrt{\mu_\text{eff} c_p(2-c_p)} \cdot \wek{\Delta}^{t}$ where \newline
          $\qquad \mu_\text{eff}=1/\left(\sum_{i=1}^\mu (w_i)^2\right)$
   \STATE $\mat{C}^{t+1} \gets (1-c_1-c_\mu)\mat{C}^t + c_1 \mat{C}^t_1 + c_\mu  \mat{C}^t_\mu$ where \newline
$\qquad \mat{C}_\mu^t=\frac{1}{\mu_\text{eff}}\sum_{i=1}^\mu w_i(\wek{d}_i^t)(\wek{d}_i^t)^\tran$, \newline
$\qquad \mat{C}_1^t=(\wek{p}_c^t)(\wek{p}_c^t)^\tran$

   \STATE $\sigma^{t+1} \gets $ CSA $(\sigma^t, \mat{C}^{t}, \wek{\Delta}^{t})$ 
      
   \STATE $t \gets t+1$
\ENDWHILE

\end{algorithmic}
\end{algorithm}




\subsection{Paradygmaty obliczeń ewolucyjnych}
\subsection{Strategie ewolucyjne}


