\newpage
\section{Wprowadzenie i cel pracy}
\label{sec:intro}

Strategie ewolucyjne są powszechnie stosowanymi algorytmami optymalizacyjnymi w przestrzeni rzeczywistoliczbowej. Wynika to przede wszystkim ze względnej prostoty mechanizmu ich działania oraz własności, które pozwalają na stosowanie ich z powodzeniem wobec wielu klas problemów optymalizacyjnych. Spośród najbardziej istotnych elementów strategii ewolucyjnych adaptacja zasięgu mutacji jest gwarantem efektywnego działania tych metod (...)


{\color{red}
    TODO: \\
    -> napisać dlaczego adaptacja zasięgu mutacji jest gwarantem \\
    -> wprowadzić ewolucje ESów od (1+1) do wielopopulacyjnych \\
}

Niniejsza praca składa się z siedmiu rozdziałów. Rozdział \ref{sec:optim} stanowi wprowadzenie do klasy problemów optymalizacji statycznej, której poświęcone są strategie ewolucyjne. W rozdziale \ref{sec:ewol} omawiany jest paradygmat obliczeń ewolucyjnych (ang. \textit{evolutionary computation}), a w szczególności -- strategii ewolucyjnych. Rozdział \ref{sec:zasieg} poświęcony jest mechanizmowi adaptacji zasięgu mutacji. Dwa kolejne rozdziały, tj. rozdział \ref{sec:modyfikacja} oraz \ref{sec:porownanie}, kolejno dotyczą wprowadzonej przeze mnie koncepcji adaptacji zasięgu mutacji oraz jej weryfikacji. Ostatni rozdział, \ref{sec:end}, stanowi podsumowanie wykonanej pracy oraz zawarte są w nim dalsze kierunki badawcze.

{\color{red}
    TODO: \\
    -> refaktoryzjaca powyższej wersji, aby brzmiała mniej debilnie \\
    
}